\documentclass[titlepage, letterpaper, 12pt, oneside]{book}
\usepackage[utf8]{inputenc}
\usepackage[english]{babel}
% --------------------------------------------------------------------------- %
% Comment these lines out to remove the watermark
% --------------------------------------------------------------------------- %
%\usepackage[printwatermark]{xwatermark}
%\newwatermark[allpages,color=red!10,angle=90,scale=3.75,xpos=60,ypos=0]
%{DRAFT\textbullet DRAFT}

% --------------------------------------------------------------------------- %
% fontspec gives problems if you compile with pdflatex
% --------------------------------------------------------------------------- %
\usepackage{fontspec}
\setmainfont{Times New Roman}
% --------------------------------------------------------------------------- %
\usepackage[letterpaper]{geometry}
\geometry{left=1.5in, right=1.25in, top=1.25in, bottom=1.25in}
\usepackage{setspace}
\usepackage{enumitem}
\usepackage{pdfpages}
%\pagenumbering{gobble}
\usepackage{multicol}
\usepackage[section]{placeins}    % Place in a section FORCEFULLY!!
\usepackage{titlesec}             % Important for changing the titles
\usepackage{blindtext}            % Testing
\usepackage{booktabs}             % I can't remember what this is for...
\usepackage{verbatim}
\usepackage{dirtytalk}            % Use \say{thing you want in quotations}
\usepackage{csquotes}             % For block quotes
\definecolor{mtsublue}{cmyk}{1.0,0.44,0,0} % Defines an MTSU blue
\usepackage[
    breaklinks=true,
    colorlinks=false,
    allbordercolors=mtsublue,
]{hyperref}
\usepackage{url}                  % For urls
\usepackage[
    labelfont=bf,
    font=small,
]{caption}                        % \caption* for non-labeled captions
\usepackage{graphicx}
\graphicspath{ {images/} }        % Put your images in the images subdirectory
\usepackage{float}
\usepackage[numbers, round]{natbib}
\bibliographystyle{plainnat}
\usepackage{todonotes}
\usepackage{caption}
\captionsetup{width = 0.7\textwidth}
\usepackage{subcaption}

\usepackage{siunitx}
\usepackage{amsmath}
\usepackage{esint}
\usepackage{mathcomp}

\usepackage{listings}
\usepackage{color}
\usepackage{dirtree}

\definecolor{dkgreen}{rgb}{0,0.6,0}
\definecolor{gray}{rgb}{0.5,0.5,0.5}
\definecolor{mauve}{rgb}{0.58,0,0.82}

\lstset{frame=tb,
    language=python,
    aboveskip=3mm,
    belowskip=3mm,
    showstringspaces=false,
    columns=flexible,
    basicstyle={\small\ttfamily},
    numbers=none,
    numberstyle=\tiny\color{gray},
    keywordstyle=\color{blue},
    commentstyle=\color{dkgreen},
    stringstyle=\color{mauve},
    breaklines=true,
    breakatwhitespace=true,
    tabsize=3,
}

\pagestyle{plain}
% --------------------------------------------------------------------------- %
% TIKZ STUFF
\usepackage{tikz}
\usetikzlibrary{
    shapes,
    arrows,
    decorations.markings,
}

% Define block styles
% --------------------------------------------------------------------------- %
\tikzstyle{startstop} = [
    rectangle,
    rounded corners,
    minimum width=3cm,
    minimum height=1cm,
    text centered,
    draw=black,
    fill=red!30,
]
\tikzstyle{io} = [trapezium,
    trapezium left angle=70,
    trapezium right angle=110,
    minimum width=3cm,
    minimum height=1cm,
    text centered,
    draw=black,
    fill=blue!30,
    text width=8cm,
]
\tikzstyle{process} = [
    rectangle,
    minimum width=3cm,
    minimum height=1cm,
    text centered,
    draw=black,
    fill=orange!30,
]
\tikzstyle{decision} = [
    diamond,
    minimum width=3cm,
    minimum height=1cm,
    text centered,
    draw=black,
    fill=green!30,
]
\tikzstyle{arrow} = [
    thick,
    ->,
    >=stealth,
]
% --------------------------------------------------------------------------- %

% --------------------------------------------------------------------------- %
% newcommands, newenvironments
% --------------------------------------------------------------------------- %
% Writing
% --------------------------------------------------------------------------- %
\newcommand{\comm}[1]{\textcolor{red}{[#1]}}    % inline comment command

% Math
% --------------------------------------------------------------------------- %
\newcommand{\pprime}{^{\prime}}
\newcommand{\sn}[2]{#1\times10^{#2}}
\newcommand{\snu}[6]{#1\times10^{#2}\,^{+\sn{#3}{#4}}_{-\sn{#5}{#6}}}
\newcommand{\snuwo}[3]{#1\,^{+#2}_{-#3}}
\newcommand{\unitsb}[1]{\,[\text{#1}]}
% --------------------------------------------------------------------------- %

% --------------------------------------------------------------------------- %
% Formatting
% --------------------------------------------------------------------------- %
\renewcommand{\contentsname}{Table of Contents}
\titleformat{\chapter}
    [hang]
    {\raggedright\bfseries\MakeUppercase}
    {\chaptertitle}
    {12pt}{}
    []
\renewcommand{\thesection}{\Roman{section}}
\renewcommand{\thesubsection}{\Alph{subsection}}
\titleformat{\section}
    %[hang]
    {\centering\bfseries\uppercase}
    {\thesection}
    {12pt}{}
    %[]
\titleformat{\subsection}
    [hang]
	{\centering\bfseries}
	{\thesubsection}
    {12pt}{}
    []

\makeatletter
\renewenvironment{thebibliography}[1]
{\section{References}% <-- this line was changed from \chapter* to \section*
	\@mkboth{\MakeUppercase\bibname}{\MakeUppercase\bibname}%
		\list{\@biblabel{\@arabic\c@enumiv}}%
			{\settowidth\labelwidth{\@biblabel{#1}}%
            \leftmargin\labelwidth
            \advance\leftmargin\labelsep
            \@openbib@code
            \usecounter{enumiv}%
            \let\p@enumiv\@empty
            \renewcommand\theenumiv{\@arabic\c@enumiv}}%
      \sloppy
      \clubpenalty4000
      \@clubpenalty \clubpenalty
      \widowpenalty4000%
      \sfcode`\.\@m}
     {\def\@noitemerr
       {\@latex@warning{Empty `thebibliography' environment}}%
      \endlist}
\makeatother

% --------------------------------------------------------------------------- %
% Author newommands
% --------------------------------------------------------------------------- %
\newcommand{\theTitle}{Title of the Paper}
\newcommand{\theTitlecaps}{\MakeUppercase{\theTitle}}
\newcommand{\theAuthor}{My Name}
\newcommand{\theInstitution}{Middle Tennessee State University}
% --------------------------------------------------------------------------- %


\begin{document}
% --------------------------------------------------------------------------- %
% The following makes equations look much nicer
% \setlength{\abovedisplayskip}{10pt}
% \setlength{\belowdisplayskip}{10pt}
% \setlength{\abovedisplayshortskip}{10pt}
% \setlength{\belowdisplayshortskip}{10pt}
% --------------------------------------------------------------------------- %

% --------------------------------------------------------------------------- %
% Front Matter
% --------------------------------------------------------------------------- %
\frontmatter
\begin{titlepage}
    \begin{center}
        \fontsize{14pt}{16.8pt}\selectfont \theTitle\\
        \vspace{48pt}
        \fontsize{12pt}{14.4pt}\selectfont
        THESIS\\
        \vspace{60pt}
        Presented to the Faculty of the Department of Physics and Astronomy\\
        in Partial Fulfillment of the Major Requirements\\
        for the Degree of\\
        \vspace{36pt}
        BACHELOR OF SCIENCE IN\\
        PHYSICS\\
        \vspace{72pt}
        \theAuthor\\
        \vspace{60pt}
        May 2018\\
        \vspace{60pt}
        \textcopyright 2018 Middle Tennessee State University\\
        All rights reserved.\\
        \vspace{12pt}
        \fontsize{9pt}{10.8pt}\selectfont
        The author hereby grants to MTSU permission to reproduce\\
        and to distribute publicly paper and electronic\\
        copies of this thesis document in whole or in part\\
        in any medium now known or hereafter created.
    \end{center}
\end{titlepage}
\thispagestyle{empty}

\clearpage

%%%%%%%%%%%%%%%%%%%%%%%%%%%%%%%%%%%%%%%%%%%%%%%%%%%%%%%%%%%%%%%%%%%%%%%%%%%%%%%%
%%%%%%%%%%%%%%%%%%%%%%%%%%%%%%%%%%%%%%%%%%%%%%%%%%%%%%%%%%%%%%%%%%%%%%%%%%%%%%%%
% Page 2
%%%%%%%%%%%%%%%%%%%%%%%%%%%%%%%%%%%%%%%%%%%%%%%%%%%%%%%%%%%%%%%%%%%%%%%%%%%%%%%%
\vspace*{24pt}
\begin{center}
    \fontsize{12pt}{14.4pt}\selectfont
    \theTitlecaps\\
    \vspace{24pt}
    \theAuthor\\
    \vspace{216pt}
\end{center}
\begin{flushleft}
    Signature of Author:
    \vspace{4pt}\hrule\vspace{4pt}
    \raggedleft
    \fontsize{10pt}{12pt}\selectfont
    Department of Physics and Astronomy\\
    May 2018\\
\end{flushleft}
\vspace{36pt}

\begin{flushleft}
    \fontsize{12pt}{14.4pt}\selectfont
    Certified by:
    \vspace{4pt}\hrule\vspace{4pt}
    \raggedleft
    \fontsize{10pt}{12pt}\selectfont
    Dr. Advisor Guy\\
    Professor of Physics \& Astronomy\\
    Thesis Supervisor\\
\end{flushleft}
\vspace{30pt}

\begin{flushleft}
    \fontsize{12pt}{14.4pt}\selectfont
    Accepted by:
    \vspace{4pt}\hrule\vspace{4pt}
    \raggedleft
    \fontsize{10pt}{12pt}\selectfont
    Dr. Ronald Henderson\\
    Professor of Physics \& Astronomy\\
    Chair, Physics \& Astronomy\\
    \vspace{36pt}
\end{flushleft}
\clearpage
\fontsize{12pt}{14.4pt}\selectfont


% --------------------------------------------------------------------------- %
% Abstract
% --------------------------------------------------------------------------- %
\section*{Abstract}
\phantomsection
\addcontentsline{toc}{section}{Abstract}
\blindtext

% --------------------------------------------------------------------------- %

% --------------------------------------------------------------------------- %
% TOC, LOF, etc.
% --------------------------------------------------------------------------- %
\tableofcontents
\listoffigures

% --------------------------------------------------------------------------- %
% Main Matter Setup
% --------------------------------------------------------------------------- %
% Setup
\mainmatter
\doublespacing

% --------------------------------------------------------------------------- %
% Content
% --------------------------------------------------------------------------- %
\section{Introduction}
\blindtext
STUFF HERE


\section{Background}
\blindtext
\Blindtext


\section{Methodology}\label{sec: methods}
\subsection{Overview}
\blindtext

\subsection{Section 2}
\blindtext

\subsection{Section 3}
\blindtext


\section{Results}
\blindtext


\section{Analysis}
\blindtext


\section{Conclusions}
\blindtext

% --------------------------------------------------------------------------- %

% --------------------------------------------------------------------------- %
% Bibliography
% --------------------------------------------------------------------------- %
\clearpage
\bibliography{mybib}
% --------------------------------------------------------------------------- %

\end{document}
